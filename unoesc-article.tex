%% Classe de documento e opções
\documentclass[%% Opções: [*] comente para remover; [>] passada para pacotes
  article,%% Tipo de documento: article, book, report, etc. [>]
  a4paper,%% Tamanho de papel: a4paper, letterpaper, etc. [>]
  12pt,%% Tamanho de fonte: 10pt, 11pt, 12pt, etc. [>]
  fleqn,%% Alinhamento de equações à esquerda (comente para centralizado) [>]
  oneside,%% Impressão: oneside (anverso) ou twoside (anverso e verso) [>]
  % twocolumn,%% Texto em duas colunas (comente para uma coluna) [>]
  chapter = TITLE,%% Títulos de capítulos em maiúsculas [*]
  section = TITLE,%% Títulos de seções (secundárias) em maiúsculas [*]
]{abntex2}

%% Pacotes utilizados
\usepackage[%% Opções
  BibURLs = false,%% Links de URLs nas referências: true ou false
  ABNTNum = none,%% Estilo numérico ABNT: none (AUTOR, ANO), dflt (1) e brkt [1]
]{unoesc-article}

\usepackage{caption}




%% Arquivo de referências
\addbibresource{unoesc-article.bib}

%% Informações do documento
%%%% Título
\titulo{Otimização da Coleta de Lixo com Tecnologias Digitais.}
%%%% Título em outro idioma
% \titleinenglish{%
%   Title of the academic work or%
%   \nextline scientific article or research project%
% }
%%%% Autor(es) e afiliação(ões)
\autor{%
  Beatriz Aparecida Miranda%
  % \thanks{%
  %   \affil{Bacharel Sistemas de Informação; UNOESC ; Chapecó}%
  %   \sep\email{samuel.silva@unoesc.edu.br}%
  % }%
  \and Emily Caroline Squena Fragoso%
  \and Isadora Costa%
  % \thanks{%
  %   \affil{Especialista em Desenvolvimento de aplicações Web; UNOPAR; Chapecó}%
  %   \sep\email{jacson.matte@unoesc.edu.br.}%
  % }%
  % \and Terceiro(a)~M.~Autor(a)%
  % \thanks{%
  %   \affil{Formação, Entidade, Cidade}%
  %   \sep\email{autor3@dominio}%
  % }%
  % \and Quarto(a)~M.~Autor(a)%
  % \thanks{%
  %   \affil{Formação, Entidade, Cidade}%
  %   \sep\email{autor4@dominio}%
  % }%
  % \and Quinto(a)~M.~Autor(a)%
  % \thanks{%
  %   \affil{Formação, Entidade, Cidade}%
  %   \sep\email{autor5@dominio}%
  % }%
}

\data{}

%% Ferramenta para criação de índices
\makeindex%
\crefname{figure}{Figura}{Figuras}
\crefname{table}{Quadro}{Quadros}
%% Início do documento
\begin{document}

\pretextual%% Elementos pré-textuais

\begin{paginadetitulo}%% Página de título

    \begin{ambienteresumo}%% Resumo
Este projeto detalha a concepção e o desenvolvimento de uma aplicação focada na gestão de resíduos sólidos para a cidade de Chapecó - SC. Diante da dificuldade de acesso a informações centralizadas sobre o descarte de lixo, a aplicação propõe ser uma solução integrada que oferece funcionalidades como consulta de horários da coleta, mapeamento interativo de pontos de entrega voluntária (PEVs) para resíduos específicos (eletrônicos, óleo, pilhas), e um guia de educação ambiental. Adicionalmente, implementa um sistema de notificações para alertar os moradores sobre a proximidade da coleta e um mecanismo de gamificação para incentivar a participação cidadã na reciclagem. O objetivo é fortalecer a cultura da sustentabilidade, aumentar as taxas de reciclagem e promover o engajamento comunitário através de uma ferramenta tecnológica acessível e interativa.

    \palavraschave{Gestão de Resíduos, Plataforma Digital, Educação Ambiental, Coleta Seletiva, Gamificação}%% Palavras-chave
    \end{ambienteresumo}
    
    % \begin{ambienteresumo}[Abstract]%% Abstract
    % \begin{otherlanguage*}{english}%% Idioma do abstract
    % The abstract text should place the work in the general context and the importance of the topic studied, briefly describe the objectives, the methodology adopted, the results obtained and the main conclusions, reporting the own contribution, in no more than 250 words.
    % It should contain neither mathematical formulas nor deductions nor bibliographical citations.
    % \palavraschave[Keywords]{word 1.\ word 2.\ word 3\ldots\ (maximum 5).}%% Keywords
    % \end{otherlanguage*}
    % \end{ambienteresumo}

\end{paginadetitulo}

\textual%% Elementos textuais
\newpage
\section{Introdução}\label{sec:intro}
O crescimento urbano acelerado da cidade de Chapecó, como em muitos centros urbanos brasileiros, traz consigo o desafio da gestão eficiente dos resíduos sólidos. A geração crescente de lixo doméstico demanda não apenas uma infraestrutura de coleta eficaz, mas também a participação ativa e consciente da população. Atualmente, informações cruciais como horários de coleta, locais de descarte para materiais específicos (eletrônicos, óleo, pilhas) e diretrizes para a separação correta do lixo encontram-se dispersas, dificultando o acesso e a adesão dos cidadãos às práticas sustentáveis.
Conforme as informações coletadas na primeira etapa do projeto e reforçadas pelas entrevistas realizadas com moradores, colegas e vizinhos, predomina uma forte sensação de incerteza e desinformação.
Entre os problemas relatados estão:
\begin{itemize}
 \•	Falta de clareza sobre os horários de coleta, especialmente nos dias alternados da coleta seletiva;
 \•	Ausência de informações consolidadas sobre os pontos de entrega voluntária (PEVs);
 \•	Dificuldade de planejamento, principalmente para moradores que saem cedo para trabalhar e não sabem o momento adequado para colocar o lixo na rua;
 \•	Confusão sobre separação correta dos resíduos, que impacta as taxas de reciclagem.
\end{itemize}

Para solucionar essa lacuna informacional e promover uma cultura de responsabilidade ambiental, este projeto propõe o desenvolvimento de uma plataforma digital. A plataforma funcionará como um canal centralizado, oferecendo aos moradores de Chapecó acesso fácil e rápido a todas as informações pertinentes à coleta de lixo, ao mesmo tempo em que educa e incentiva o engajamento através de funcionalidades interativas e de gamificação.

\section{Delimitação do Tema e Justificativa}\label{ssec:teor}
\subsection{Delimitação}
O presente projeto foca no desenvolvimento de uma solução tecnológica (plataforma digital) para otimizar a comunicação e a gestão de resíduos sólidos domésticos na área urbana de Chapecó, Santa Catarina. O escopo abrange a divulgação de cronogramas de coleta de lixo orgânico e seletivo, o mapeamento de pontos de entrega voluntária (PEVs) para recicláveis e resíduos especiais (lixo eletrônico, pilhas, óleo de cozinha), a criação de um canal para denúncias de descarte irregular e a implementação de um sistema de notificações e gamificação para estimular a participação cidadã.

\subsection{Justificativa}
A gestão inadequada de resíduos sólidos representa um dos maiores desafios ambientais e de saúde pública para os municípios. Em Chapecó, a falta de uma ferramenta unificada que oriente o cidadão sobre o descarte correto contribui para uma menor taxa de reciclagem, o descarte incorreto de materiais perigosos e a formação de pontos de lixo irregulares, que sobrecarregam os serviços de limpeza urbana e degradam o meio ambiente.
A criação de uma plataforma digital atende a uma necessidade real da população por informações claras e acessíveis. Ao utilizar a tecnologia, que possui alta penetração na sociedade, o projeto torna-se uma ferramenta de grande alcance e baixo custo para a educação ambiental. Além de informar, a aplicação busca transformar o cidadão em um agente ativo na fiscalização e na promoção de um ambiente urbano mais limpo e sustentável. O incentivo através da gamificação tem o potencial de criar um engajamento contínuo, transformando a prática da reciclagem em um hábito positivo e coletivo, fortalecendo o senso de comunidade e responsabilidade compartilhada.


\newpage
\section{Objetivo}
Desenvolver uma plataforma digital que facilite o acesso dos cidadãos de Chapecó às informações sobre a coleta de resíduos, promova a educação ambiental sobre o descarte correto e incentive o engajamento da população em práticas de reciclagem por meio de ferramentas interativas de mapeamento, notificação e gamificação.


\section{Trabalhos Relacionados}
Para compreender o cenário atual da gestão digital de resíduos e identificar oportunidades de inovação, foram analisados sistemas e soluções existentes no mercado nacional e internacional. As principais plataformas estudadas foram:

\subsection{Sensoneo – Driver App e Route Planning}
A Sensoneo oferece uma solução completa voltada à otimização da coleta de resíduos, incluindo:
\begin{itemize}
  
    \ •	Rotas pré-planejadas e otimizadas, reduzindo consumo de combustível e tempo de operação;
    \ •	Navegação guiada por voz, que facilita a adaptação de novos motoristas;
    \ •	Atualização em tempo real de rotas, considerando ruas estreitas e restrições de veículos;
    \•	Gestão eficiente de grandes volumes de dados, incluindo milhares de lixeiras e veículos, minimizando erros em processos manuais.Esses sistemas buscam reduzir custos operacionais e impactos ambientais, como ruídos e congestionamentos urbanos .
\end{itemize}

\subsection{Sistemas Municipais de Monitoramento (Ex.: Vitória – ES)}
A prefeitura de Vitória utiliza monitoramento via GPS para acompanhar os caminhões de coleta, permitindo maior transparência, agilidade e controle do serviço. O sistema reduz falhas operacionais e possibilita respostas mais rápidas a reclamações de moradores,

\subsection{Plataformas Brasileiras de Gestão de Resíduos}
Algumas empresas oferecem soluções voltadas para gestão corporativa ou municipal, tais como:
\begin{itemize}
    \ •	MeuResíduo;
    \ • Zero Route Planner;
    \•	DigiResíduos;
\end{itemize}
Essas plataformas fornecem recursos para rastreamento de resíduos, logística reversa e relatórios ambientais, porém não oferecem foco direto no cidadão comum, nem funcionalidades integradas como gamificação, notificações personalizadas ou mapeamento amigável de PEVs 

\subsection{Pontos Fortes e Limitações das Soluções Existentes}
| **Solução**                           | **Pontos Fortes**                                                                 | **Limitações**                                                                                 |
|---------------------------------------|-----------------------------------------------------------------------------------|-------------------------------------------------------------------------------------------------|
| Sensoneo                              | Otimização avançada de rotas; redução de emissões; uso intensivo de tecnologia   | Voltado à gestão municipal, **não ao cidadão**                                                  |
| Monitoramento por GPS (Vitória-ES)    | Transparência; acompanhamento em tempo real                                       | Sem aplicativo para moradores; não informa horários personalizados                             |
| MeuResíduo / DigiResíduos             | Boa gestão institucional; rastreamento                                            | Foco corporativo; ausência de mapas interativos ou gamificação                                 |
| Zero Route Planner                    | Planejamento eficiente                                                             | Não é voltado à coleta urbana doméstica                                                         |

\section{Diferenciais da Plataforma Proposta }
Com base nas pesquisas e entrevistas, o projeto se diferencia por:
\begin{itemize}
 \•	Foco direto no cidadão, não apenas nas prefeituras.
 \•	Centralização total das informações, eliminando a dispersão atual indicada nas entrevistas.
 \•	Sistema de notificações personalizadas, resolvendo a principal dor identificada: a incerteza dos horários da coleta.
 \•	Mapeamento amigável dos PEVs, que falta nos sistemas atuais.
 \•	Gamificação, inexistente nas plataformas analisadas.
 \•	Ferramentas de educação ambiental, melhorando o entendimento sobre separação do lixo.
 \•	Apoio ao planejamento diário do morador, levando em consideração relatos de que muitos saem cedo para trabalhar e não sabem o momento correto de colocar o lixo na rua.
 
\end{itemize}
   
\section{Propostas do Sistema}
\subsection{Requisitos Funcionais}
\begin{itemize}
  
    \item RF001: O sistema deve permitir que o usuário cadastre seu endereço para personalizar as informações de coleta.
    \item RF002: O sistema deve exibir os dias e horários da coleta seletiva e da coleta convencional com base no endereço cadastrado.
    \item RF003: O sistema deve fornecer informações sobre coletas para resíduos específicos (eletrônicos, óleo, pilhas)
    \item RF04: O sistema deve possuir uma seção de "Guia de Reciclagem" com informações sobre como separar corretamente os resíduos.
    \item RF005: O sistema deve implementar uma ferramenta de busca onde o usuário pode digitar o nome de um item e receber instruções sobre o descarte correto.
    \item RF006: O sistema deve exibir a localização dos PEVs no mapa, identificados por ícones.
    \item RF007: O sistema deve, ao selecionar um PEV, exibir informações detalhadas (ex: nome, endereço completo, horário de funcionamento, tipos de resíduos aceitos).
    \item RF008: O sistema deve permitir ao usuário filtrar os PEVs por tipo de resíduo (ex: pilhas e baterias, óleo de cozinha, eletrônicos, lâmpadas).
    \item RF008: O sistema deve permitir ao usuário traçar uma rota da sua localização atual até um PEV selecionado (integração com apps de mapa como Google Maps).
    \item RF009: O sistema deve enviar uma notificação push ao usuário para lembrá-lo da proximidade do horário da coleta.
    \item RF010: O sistema deve permitir ao usuário configurar (ligar/desligar) os tipos de notificações que deseja receber.
    \item RF011: O sistema deve notificar o usuário sobre mudanças emergenciais na coleta (ex: feriados, problemas operacionais).
    \item RF012: O sistema deve permitir o "check-in" (via GPS/QR Code) em um PEV para validar o descarte e ganhar pontos.
    \item RF013: O sistema deve conceder medalhas (badges) ou conquistas por marcos alcançados (ex: "Primeiro Descarte de Pilhas").
    \item RF014: O sistema deve possuir um painel administrativo para que os responsáveis possam atualizar os horários de coleta por bairro/zona.
    \item RF015: O sistema deve permitir ao administrador gerenciar (adicionar, editar, remover) os PEVs no mapa.
    \item RF016: O sistema deve permitir ao administrador enviar notificações em massa para os usuários.

\end{itemize}

\subsection{Requisitos Não Funcionais}
\begin{itemize}
  \item RNF001: A interface do usuário (UI) deve ser intuitiva e limpa. 
  \item RNF002: O layout deve ser responsivo, adaptando-se corretamente a diferentes tamanhos e densidades de tela de smartphones.
  \item RNF003: O sistema deve seguir as diretrizes de acessibilidade (WCAG), permitindo o uso por pessoas com deficiências (ex: contraste de cores adequado, fontes legíveis, compatibilidade com leitores de tela).
  \item RNF004: O aplicativo deve lidar bem com a perda de conexão, informando ao usuário e, se possível, mantendo o acesso a dados offline (ex: guia de reciclagem já baixado).
  \item RNF005: Todos os dados de usuário (endereço, e-mail, etc.) devem ser armazenados de forma criptografada no banco de dados.
  \item RNF006: Toda a comunicação entre o aplicativo e os servidores deve ser feita via HTTPS.
  \item RNF007: O sistema deve estar em conformidade com a LGPD (Lei Geral de Proteção de Dados), solicitando consentimento claro para o uso de dados de localização e notificações.
\end{itemize}

\section{Referencias}
\begin{itemize}
  \SENSONEO. Driver App. Disponível em: https://sensoneo.com/products/driver-app/
  \SENSONEO. Route Planning. Disponível em: https://sensoneo.com/products/route-planning/
  \VITÓRIA (ES). Prefeitura Municipal. Monitoramento da coleta de lixo pelo sistema GPS agiliza o serviço na capital. Disponível em:
https://www.vitoria.es.gov.br/noticias/monitoramento-da-coleta-de-lixo-pelo-sistema-gps-agiliza-o-servico-na-capital-11229
  \ZERO ROUTE PLANNER. Gestão de resíduos. Disponível em:https://zeorouteplanner.com/pt/gest%C3%A3o-de-res%C3%ADduos/
  \MEU RESÍDUO. Plataforma de gestão de resíduos. Disponível em:https://www.meuresiduo.com/
  \DIGIRESÍDUOS. Sistema de gestão de resíduos sólidos. Disponível em:https://www.digiresiduos.com.br/
\end{itemize}

\postextual%% Elementos pós-textuais
\newpage
%%\printbibliography%% Referências

%% Fim do documento
\end{document}
